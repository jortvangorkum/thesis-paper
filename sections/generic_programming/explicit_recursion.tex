\section{Explicit recursion}
\label{sec-explicit-recursion}

% There are two ways to define the representation of values. Those that have information about which fields of the constructors of the datatype in question are recursive versus those that do not.

The previous implementation of the \texttt{length} function is implemented for a shallow representation. A shallow representation means that the recursion of the datatype is not explicitly marked. Therefore, we can only convert one layer of the value into a generic representation using the \texttt{from} function. 

Alternatively, by marking the recursion of the datatype explicitly, also called the deep representation, the entire value can be converted into a generic representation in one go. To mark the recursion, a fixed-point operator (\texttt{Fix}) is introduced. Then, using the fixed-point operator we can define a \texttt{from} function that given the pattern functors have an instance of \texttt{Functor}\footnote{The \texttt{Functor} instances for the pattern functors can be found in Section \ref{app-inst-functor-patfun}}, return a generic representation of the entire value.

\begin{minted}{haskell}
data Fix f = In { unFix :: f (Fix f) }

deepFrom :: (Regular a, Functor (PF a)) => a -> Fix (PF a)
deepFrom = In . fmap deepFrom . from
\end{minted}

Subsequently, we can define a \texttt{cata} function which can use the explicitly marked recursion by applying a function at every level of the recursion. Then using the \texttt{cata} function we can define the same \texttt{length} function as in the previous section, but just in a single line. However, this deep representation does come at the cost that the implementation is less efficient than the shallow representation. 

\begin{minted}{haskell}
cata :: Functor f => (f a -> a) -> Fix f -> a
cata = f . fmap (cata f) . unFix

length' :: (Regular a, GLength (PF a), Functor (PF a), Foldable (PF a)) 
        => a -> Int
length' = cata ((1+) . sum) . deepFrom
\end{minted}