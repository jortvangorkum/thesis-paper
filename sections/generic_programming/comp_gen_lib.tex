\section{Comparison Generic Libraries}
\label{sec-comp-gen-libs}

There are a multitude of generic programming libraries in Haskell. Choosing between the generic programming libraries is an assessment between different criteria. The criteria for a generic programming library for this paper are
\begin{enumerate*}[label=(\alph*)]
    \item the generic representation,
    \item support of mutually recursive datatypes and
    \item ease of use.
\end{enumerate*}

\begin{table}[H]
    \centering
    \begin{tabular*}{\textwidth}{@{\extracolsep{\fill}}|l c c c|} 
        \hline
        \textbf{Library} & \textbf{Representation} & \textbf{Mutually Recursive} & \textbf{Ease of Use} \\ 
        \hline
        \texttt{regular}\cite{regular2022} & Pattern Functor & $\times$ & Easy \\ 
        \hline
        \texttt{multirec}\cite{multirec2022} & Pattern Functor & $\checkmark$ & Medium \\
        \hline
        \texttt{generics-sop}\cite{genericssop2022} & SOP & $\times$ & Difficult \\
        \hline 
        \texttt{generics-mrsop}\cite{genericsmrsop2022} & SOP & $\checkmark$ & Complex \\
        \hline
    \end{tabular*}
    \caption{Criteria overview of generic programming libraries in Haskell}
\end{table}

\question{How to compare the different libraries based on complexity?}

The \texttt{regular} library is the easiest to understand generic programming library. This is because the library uses pattern functors and does not support mutually recursive datatypes.  

% More complex universe, does support mutually recursive datatypes, but it becomes more complex.


% The paper \citetitle{rodriguez2008comparing}\cite{rodriguez2008comparing} by \citeauthor{rodriguez2008comparing} also makes a comparison of the current landscape of generic programming libraries in \citeyear{rodriguez2008comparing}. 