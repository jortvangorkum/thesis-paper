\chapter{Datatype-Generic Programming}

The implementation in Chapter \ref{chap-spec-impl} is an efficient implementation for incrementally computing the summation over a \texttt{Tree} datatype. However, when we want to implement this functionality for a different datatype, a lot of code needs to be copied while the process remains the same. This results in poor maintainability, is error-prone and is in general boring work.

An example of reducing manual implementations for datatypes is the \textit{deriving} mechanism in Haskell. The built-in classes of Haskell, such as \texttt{Show, Ord, Read}, can be derived for a large class of datatypes. However, deriving is not supported for custom classes. Therefore, we use \textit{Datatype-Generic Programming}\cite*{gibbons2006datatype} to define functionality for a large class of datatypes.

In this chapter, we introduce Datatype-Generic Programming, also known as \textit{generic programming} or \textit{generics} in Haskell, as a technique that exploits the structure of datatypes to define functions by induction over the type structure. This prevents the need to write the previously defined functionality for every datatype.

\section{Introduction}
There are multiple generic programming libraries, however to demonstrate the workings of generic programming we will be using a single library as inspiration, named \texttt{regular}\cite*{regular2022}. Here the generic representation of a datatype is called a \textit{pattern functor}. A pattern functor is a stripped-down version of a data type, by only containing the constructor but not the recursive structure. The recursive structure is done explicitly by using a fixed-point operator. 

First, the pattern functors defined in \texttt{regular} are 5 core pattern functors and 2 meta information pattern functors. The core pattern functors describe the datatypes. The meta information pattern functors only contain information (e.g., constructor name) but not any structural information.

\begin{minted}{haskell}
data U r         = U                 -- Empty constructors
data I r         = I r               -- Recursive call
data K a r       = K a               -- Constants
data (f :+: g) r = L (f r) | R (g r) -- Sums (Choice)
data (f :*: g) r = (f r) :*: (g r)   -- Products (Combine)
\end{minted}

The conversion from regular datatypes into pattern functors is done by the \texttt{Regular} type class. The \texttt{Regular} type class has two functions. The \texttt{from} function converts the datatype into a pattern functor and the \texttt{to} function converts the pattern functor back into a datatype. In \texttt{regular}, the pattern functor is represented by a type family. Then using the \texttt{Regular} conversion to a pattern functor, we can write the \texttt{Tree} datatype from Chapter \ref{chap-spec-impl} as:

\begin{minted}{haskell}
type family PF a :: * -> *

class Regular a where
    from :: a -> PF a a
    to   :: PF a a -> a

type instance PF (Tree a) = K a                -- Leaf
                         :+: (I :*: K a :*: I) -- Node
\end{minted}

To demonstrate the workings of generic programming, we are going to implement a simple generic function which determines the length of an arbitrary datatype. First, we define the length function within a type class. The type class is used, to define how to calculate the length for every pattern functor \texttt{f}. 

\begin{minted}{haskell}
class GLength f where
    glength :: (a -> Int) -> f a -> Int
\end{minted}

Writing instances for the empty constructor \texttt{U} and the constants \texttt{K} is simple because both pattern functors return zero. The \texttt{U} pattern functor returns zero, because it does not contain any children. The \texttt{K} pattern functor returns zero, because we do not count constants for the length. 

\begin{minted}{haskell}
instance GLength U where
    glength _ _ = 0

instance GLength (K a) where
    glength _ _ = 0
\end{minted}

The instances for sums and products pattern functors are quite similar. The sums pattern functor recurses into the specified choice. The product pattern functor recurses in both constructors and combines them.

\begin{minted}{haskell}
instance (GLength f, GLength g) => GLength (f :+: g) where
    glength f (L x) = glength f x
    glength f (R x) = glength f x

instance (GLength f, GLength g) => GLength (f :*: g) where
    glength f (x :*: y) = glength f x + glength f y
\end{minted}

The instance for the recursive call \texttt{I} needs an additional argument. Because, we do not know the type of \texttt{x}, so an additional function (\inlinehaskell{f :: a -> Int}) needs to be given which converts \texttt{x} into the length for that type.

\begin{minted}{haskell}
instance GLength I where
    glength f (I x) = f x
\end{minted}

Then using the \texttt{GLength} instances for all pattern functors, a function can be defined using the generic length function. By first, converting the datatype into a pattern functor representation, then calling \texttt{glength} given recursively itself, and for every recursive call increase the length by one. 

\begin{minted}{haskell}
length :: (Regular a, GLength (PF a)) => a -> Int
length = 1 + glength length (from x)
\end{minted}

\begin{minted}{haskell}
> length [1, 2, 3]
    3
> length (Node (Leaf 1) 2 (Leaf 3))
    3
> length (Map.fromList [("1", 1), ("2", 2), ("3", 3)])
    3
\end{minted}

\subsection{Explicit recursion}
\label{sec-explicit-recursion}

% There are two ways to define the representation of values. Those that have information about which fields of the constructors of the datatype in question are recursive versus those that do not.

The previous implementation of the \texttt{length} function is implemented for a shallow representation. A shallow representation means that the recursion of the datatype is not explicitly marked. Therefore, we can only convert one layer of the value into a generic representation using the \texttt{from} function. 

Alternatively, by marking the recursion of the datatype explicitly, also called the deep representation, the entire value can be converted into a generic representation in one go. To mark the recursion, a fixed-point operator (\texttt{Fix}) is introduced. Then, using the fixed-point operator we can define a from function that given the pattern functors have an instance of Functor\footnote{The Functor instances for the pattern functors can be found in Section \ref{app-inst-functor-patfun}}, return a generic representation of the entire value.

\begin{minted}{haskell}
data Fix f = In { unFix :: f (Fix f) }

deepFrom :: (Regular a, Functor (PF a)) => a -> Fix (PF a)
deepFrom = In . fmap deepFrom . from
\end{minted}

Subsequently, we can define a \texttt{cata} function which can use the explicitly marked recursion by applying a function at every level of the recursion. Then using the \texttt{cata} function we can define the same \texttt{length} function as in the previous section, but just in a single line. However, this deep representation does come at the cost that the implementation is less efficient than the shallow representation. 

\begin{minted}{haskell}
cata :: Functor f => (f a -> a) -> Fix f -> a
cata = f . fmap (cata f) . unFix

length' :: (Regular a, GLength (PF a), Functor (PF a), Foldable (PF a)) 
        => a -> Int
length' = cata ((1+) . sum) . deepFrom
\end{minted}

\subsection{Sums of products}
\todo[inline]{Describe the differences between defining generic data types}

\subsection{Mutually recursive datatypes}
\todo[inline]{Describe what mutually recursive datatypes are and why do we need to know about it}

\section{Comparison Generic Libraries}
\citetitle{rodriguez2008comparing}\cite{rodriguez2008comparing}