\begin{abstract}
\todo[inline]{Make the abstract smaller}
Incremental computation attempts improving performance when the same computation is performed with slightly different input. A specific technique of incremental computation is memoization. Memoization stores the result of a computation and returns the cached result when the same input occurs again. As a result, a large part of memoization becomes dependent on determining if the input is equal to an already cached input. So, when a computation is given a large recursive data structure, the entire data structure needs to be traversed through to determine if it has a cached result. This is inefficient, so, to improve the performance of memoization this paper introduces an incremental algorithm which determines the equality in constant time. This is accomplished by storing hash values/digests, which describe the internal structure, inside the data structure. Furthermore, the incremental algorithm describes how to efficiently update the digests when the data structure changes using a Zipper. The incremental algorithm is also implemented using Datatype-generic programming, to support the class of regular datatypes. At the same time, the usage of the generic implementation stays the same for  the developers as writing the non-incremental algorithm in Haskell. In the end, we show that the performance is better than the non-incremental version with minimal extra memory usage, when correctly tuned with cache policies.
\end{abstract}

% Word limit: 150 - 200 words