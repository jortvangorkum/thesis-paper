\chapter{Specific Implementation}
\label{chap-spec-impl}

\begin{minted}{haskell}
data Tree a = Leaf a
            | Node (Leaf a) a (Leaf a)
\end{minted}

\begin{minted}{haskell}
sumTree :: Tree Int -> Int
sumTree (Leaf x)     = x
sumTree (Node l x r) = x + (sumTree l) + (sumTree r)
\end{minted}

Computing a value of a data structure can easily be defined in Haskell, but every time there is a small change in the \texttt{Tree}, the entire \texttt{Tree} needs to be recomputed. This is inefficient, because most of the computations have already been performed in the previous computation. 

To prevent recomputation of already computed values, the technique memoization is introduced. Memoization is a technique where the results of computational intensive tasks are stored and when the same input occurs, the result is reused. 

The comparison of two values in Haskell is done with the \texttt{Eq} typeclass, which implements the equality operator \inlinehaskell{(==) :: a -> a -> Bool}. So, an example implementation of the \texttt{Eq} typeclass for the \texttt{Tree} datatype would be:

\begin{minted}{haskell}
instance Eq a => Eq (Tree a) where
    Leaf x1       == Leaf x2       = x1 == x2
    Node l1 x1 r1 == Node l2 x2 r2 = x1 == x2 && l1 == l2 && r1 == r2
    _             == _             = False
\end{minted}

The problem with using this implementation of the \texttt{Eq} typeclass for Memoization is that for every comparison of the \texttt{Tree} datatype the equality is computed. This is inefficient because the equality implementation has to traverse the complete \texttt{Tree} data structure to know if the \texttt{Tree}'s are equal. 

To efficiently compare the \texttt{Tree} datatypes, we need to represent the structure in a manner which does not lead to traversing to the complete \texttt{Tree} data structure. This can be accomplished using a \texttt{hash} function. A hash function is a process of transforming a data structure into an arbitrary fixed-size value, where the same input always generates the same output. 

One of the disadvantages of using hashes is \textit{hash collisions}. Hash collisions happen when two different pieces of data have the same hash. This is because a hash function has a limited amount of bits to represent every possible combination of data. Using the formula $p = \epsilon^{\frac{-k(k-1)}{2N}}$ from \citetitle{hashcoll2011}\cite{hashcoll2011}, we can calculate a $50\%$ chance of getting a hash collision with a collection of $k$. The hash function CRC-32 needs a collection of 77163 hash values. The hash function MD5 needs a collection of $\num{5.06e9}$ hash values. And, the hash function SHA-1 needs a collection of $\num{1.42e24}$ hash values. As a result of, we can say that for most popular hash functions, hash collisions are negligible.

\begin{minted}{haskell}
class Hashable a where
    hash :: a -> Hash

instance Hashable a => Hashable (Tree a) where
    hash (Leaf x)     = concatHash [hash "Leaf", hash x]
    hash (Node l x r) = concatHash [hash "Node", hash x, hash l, hash r]
\end{minted}

The hashes can then be used to efficiently compare two \texttt{Tree} data structures, without having to traverse the entire \texttt{Tree} data structure. To keep track of the intermediate results of the computation, we store the results in a \texttt{Map}. A \texttt{Map}, also known as a dictionary, is an implementation of mapping a key to a value. In our next example the \texttt{Hash} is the key and the value is the intermediate result.

\begin{minted}{haskell}
sumTreeInc :: Tree Int -> (Int, Map Hash Int)
sumTreeInc l@(Leaf x)     = (x, insert (hash l) x empty)
sumTreeInc n@(Node l x r) = (y, insert (hash n) y (ml <> mr))
    where
        y = x + xl + xr
        (xl, ml) = sumTreeInc l
        (xr, mr) = sumTreeInc r
\end{minted}

Then after the first computation over the entire \texttt{Tree}, we can recompute the \texttt{Tree} using the previously created \texttt{Map}. Thus, when we recompute the \texttt{Tree}, we first look in the \texttt{Map} if the computation has already been performed then return the result. Otherwise, compute the result and store it in the \texttt{Map}.

\question{Maybe add the more efficient implementation of merging maps?}
\begin{minted}{haskell}
sumTreeIncMap :: Map Hash Int -> Tree Int -> (Int, Map Hash Int)
sumTreeIncMap m l@(Leaf x) = case lookup (hash l) m of
    Just x  -> (x, m) 
    Nothing -> (x, insert (hash l) x empty)
sumTreeIncMap m n@(Node l x r) = case lookup (hash n) m of
    Just x  -> (x, m)
    Nothing -> (y, insert (hash n) y (ml <> mr))
        where
            y = x + xl + xr
            (xl, ml) = sumTreeIncMap m l
            (xr, mr) = sumTreeIncMap m r
\end{minted}

Generating a hash for every computation over the data structure is time-consuming and unnecessary, because most of the \texttt{Tree} data structure stays the same. The work of \citeauthor{miraldo2019efficient}\cite{miraldo2019efficient} inspired the use of the Merkle Tree. A Merkle Tree is a data structure which integrates the hashes within the data structure.

\section{Merkle Tree (\texttt{TreeH})}
First we introduce a new datatype \texttt{TreeH}, which contains a \texttt{Hash} for every constructor in \texttt{Tree}. Then to convert the \texttt{Tree} datatype into the \texttt{TreeH} datatype, the structure of the Tree is hashed and stored into the datatype using the \texttt{merkle} function.

\begin{minted}{haskell}
data TreeH a = LeafH Hash a
             | NodeH Hash (Leaf a) a (Leaf a)
\end{minted}

\begin{minted}{haskell}
merkle :: Tree Int -> TreeH Int
merkle l@(Leaf x) = LeafH (hash l) x
merkle (Node l x r) = NodeH h l' x r'
    where
        h = hash ["Node", x, getHash l', getHash r']
        l' = merkle l
        r' = merkle r
\end{minted}

The precomputed hashes can then be used to easily create a \texttt{Map}, without computing the hashes every time the \texttt{sumTreeIncH} function is called.

\begin{minted}{haskell}
sumTreeIncH :: TreeH Int -> (Int, Map Hash Int)
sumTreeIncH (LeafH h x)     = (x, insert h x empty)
sumTreeIncH (NodeH h l x r) = (y, insert h y (ml <> mr))
    where
        y = x + xl + xr
        (xl, ml) = sumTreeInc l
        (xr, mr) = sumTreeInc r
\end{minted}

The problem with this implementation is, that when the \texttt{Tree} datatype is updated, the entire \texttt{Tree} needs to be converted into a \texttt{TreeH}, which is linear in time. This can be done more efficiently, by only updating the hashes which are impacted by the changes. Which means that only the hashes of the change and the parents need to be updated. 

The first intuition to fixing this would be using a pointer to the value that needs to be changed. But because Haskell is a functional programming language, there are no pointers. Luckily, there is a data structure which can be used to efficiently update the data structure, namely the Zipper\cite{huet1997zipper}.

\section{Zipper}

% \question{Add a visual example?}

The Zipper is a technique of representing a data structure by keeping track of how the data structure is being traversed through. The Zipper was first described by \citeauthor{huet1997zipper}\cite{huet1997zipper} and is a solution for efficiently updating pure recursive data structures in a purely functional programming language (e.g., Haskell). This is accomplished by keeping track of the downward current subtree and the upward path, also known as the \textit{location}. 

To keep track of the upward path, we need to store the path we traverse to the current subtree. The traversed path is stored in the \texttt{Cxt} datatype. The \texttt{Cxt} datatype represents three options the path could be at: the \texttt{Top}, the path has traversed to the left (\texttt{L}), or the path has traversed to the right (\texttt{R}).

\begin{minted}{haskell}
data Cxt a = Top
           | L (Cxt a) (Tree a) a
           | R (Cxt a) (Tree a) a

type Loc a = (Tree a, Cxt a)

enter :: Tree a -> Loc a
enter t = (t, Top)           
\end{minted}

Using the \texttt{Loc}, we can define multiple functions on how to traverse through the \texttt{Tree}. Then, when we get to the desired location in the \texttt{Tree}, we can call the \texttt{modify} function to change the \texttt{Tree} at the current location.

\begin{minted}{haskell}
left :: Loc a -> Loc a
left (Node l x r, c) = (l, L c r x)

right :: Loc a -> Loc a
right (Node l x r, c) = (r, R c l x)

up :: Loc a -> Loc a
up (t, L c r x) = (Node t x r, c)
up (t, R c l x) = (Node l x t, c)

modify :: (Tree a -> Tree a) -> Loc a -> Loc a
modify f (t, c) = (f t, c)
\end{minted}

Eventually, when every value in the \texttt{Tree} has been changed, the entire \texttt{Tree} can then be rebuilt using the \texttt{Cxt}. By recursively calling the \texttt{up} function until the top is reached, the current subtree gets rebuilt. And when the top is reached, the entire tree is then returned.

\begin{minted}{haskell}
leave :: Loc a -> Loc a
leave l@(t, Top) = l
leave l = top (up l)
\end{minted}

\subsection{Zipper \texttt{TreeH}}

The implementation of the Zipper for the \texttt{TreeH} datatype is the same as for the \texttt{Tree} datatype. However, the \texttt{TreeH} also contains the hash of the current and underlying data structure. Therefore, when a value is modified in the \texttt{TreeH}, all the parent nodes of the modified value needs to be updated. 

The \texttt{updateLoc} function modifies the value at the current location, then checks if the location has any parents. If the location has any parents, go up to that parent, update the hash of that parent and recursively update the parents hashes until we are at the top of the data structure. Otherwise, return the modified locations, because all the other hashes are not affected by the change. 

\begin{minted}{haskell}
updateLoc :: (TreeH a -> TreeH a) -> Loc a -> Loc a
updateLoc f l = if top l' then l' else updateParents (up l')
    where
        l' = modify f l

        updateParents :: Loc a -> Loc a
        updateParents (Loc x Top) = Loc (updateHash x) Top
        updateParents (Loc x cs)  = updateParents $ up (Loc (updateHash x) cs)
\end{minted}

Then, the \texttt{update} function can be defined using the \texttt{updateLoc} function, by first traversing through the data structure with the given directions. Then modifying the location using the \texttt{updateLoc} function and then leave the location and the function results in the updated data structure.

\begin{minted}{haskell}
update :: (TreeH a -> TreeH a) -> [Loc a -> Loc a] -> TreeH a -> TreeH a 
update f dirs t = leave $ updateLoc f l'
    where
        l' = applyDirs dirs (enter t)
\end{minted}