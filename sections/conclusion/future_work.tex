\section{Future Work}

The Thesis paper still has some topics that need further exploring. A few small topics would be: 
\begin{itemize}
  \item Implement a storage medium which does not rehash the key (because the key is already a hash).
  \item Implement a default cache replacement policy which generally works well for all algorithms. 
  \item Implement the generation of the pattern synonyms using \texttt{TemplateHaskell}.
  \item Run the benchmarks in a more stable environment than a laptop.
  \item Benchmark the results against the \texttt{MemoTrie} package.
\end{itemize}

\subsection{Support for Mutually Recursive Datatypes}
The current implementation of the algorithm only supports regular datatypes. This makes the possible datatypes this algorithm can be used with quite limited. To increase the amount of datatypes, we need to support mutually recursive datatypes. One big advantage of supporting mutually recursive datatypes is that then most abstract syntax tree (AST) of popular programming languages can be used with the algorithm. So, for example, we can incrementally calculate the cyclomatic complexity metric over the AST of a programming language. 

\subsection{Implement the algorithm using Sums-of-Products}
The generic implementation of the algorithm uses pattern functors. Pattern functors are a simple way to define generic functionality. However, the pattern functors have no restrictions on how they are combined. The Sums-of-Products represents the Haskell datatype better than the pattern functors, by only limiting the creation of sums of products. This can make it easier to implement the generic version of the algorithm or add additional optimizations.

\subsection{Benchmarking with real-world data}
The current results presented in this paper are all synthetic benchmarks. This makes it easier to compare the results and make conclusions. However, it does not represent how well the algorithms actually perform in the real-world. A real-world example would be to compute a metric over public available code\footnote{This does mean that the algorithm first needs to support mutually recursive datatypes.}.

\subsection{Support for a new input without changes}
The incremental algorithm presented in this paper expects that the changes are given to them by an external system/process (e.g., a structure editor). This limits the way this algorithm can be used by other developers. To support more use-cases (e.g., a compiler), the incremental algorithm needs to support that when given a new input, without changes, it can still efficiently compute the result. A way to support this, is to perform a \textit{diff} algorithm between the previous input and the current input and compute the changes between the two. Then using these changes update the previous input and compute the result using the incremental algorithm. An implementation of such \textit{diff} algorithm can be found in the paper \citetitle*{bransen2015incremental} by \citeauthor{bransen2015incremental} in Section 4.6.3. However, this is certainly not faster than the current incremental algorithm and could be quite slower than using the non-incremental algorithm. 

\subsection{Prioritization}
The simplest topic to research which adds a lot of value to this project is the generation of pattern synonyms using \texttt{TemplateHaskell}. Currently, the user has to define their own pattern synonyms which is quite developer unfriendly. The learning curve of \texttt{TemplateHaskell} is quite steep, but the eventual implementation is not complex. An example of using \texttt{TemplateHaskell} to generate pattern synonyms can be found at \href{https://hackage.haskell.org/package/generics-mrsop-2.3.0/docs/Generics-MRSOP-TH.html}{\textit{Generics-MRSOP-TH}}\cite*{genericsmrsop2022}.

A more complex topic which adds a lot of value to the project is supporting mutually recursive datatypes. By supporting mutually recursive datatypes, the incremental algorithm can then be used for AST's of most programming languages. As a result, a new generic programming library needs to be used which supports mutually recursive datatypes (e.g., \texttt{multirec}\cite*{multirec2022}). This can take a lot of time, 