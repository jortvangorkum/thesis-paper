\chapter{Conclusion}

% The results show that the incremental computation is faster, than the non-incremental computation after $10^3$ nodes. However, the initial pass of the incremental computation is slower than the non-incremental computation. Therefore, the optimal use case for the incremental computation is one where there are a lot of small updates to the data structure. This can be accomplished with almost no additional code implemented by the user. 

% Furthermore, the additional memory overhead needed for keeping track of all the intermediate results is relatively low. This makes the use of the incremental computation not require a system with additional memory available compared to using the non-incremental computation. 

% Nonetheless, if memory usage becomes an issue overtime. A cache replacement policy can be chosen to be used. 

Setup of conclusion:
\begin{itemize}
  \item Start with the goal/contributions of the paper.
  \item Show that the results are successful
  \item Conclude with a one-liner.
\end{itemize}

\todo[inline]{Write overview which cache policy replacement for which use case}

\section{Future work}
\todo[inline]{Write future work}

\begin{itemize}
  \item Only results for a sum over data structure computation (no real-world results / only synthetic results)
  \item Only support regular datatypes (not mutually recursive datatypes)
  \item Does not use Sums-of-Products
  \item 
\end{itemize}