\section{Generic Zipper}

For the implementation of a generic zipper, we need to define
\begin{enumerate*}[label=(\Alph*)]
  \item a datatype which keeps track of the location inside the data structure,
  \item a context which keeps track of the locations which have been traversed through
  \item and functions which facilitate traversing through the data structure.
\end{enumerate*}
The implementation of the general zipper is based on the paper \citetitle{bransen2013generic} by \citeauthor{bransen2013generic}\cite{bransen2013generic}.

The context (\texttt{Ctx}) is implemented using a type family\footnote{``\textit{Type families are to vanilla data types what type class methods are to regular functions.}''\cite{haskellwiki2022typefamilies}}. Type families can be defined in two manners, standalone or associated with a type class. For the explanation we use the standalone definition because the code is less clumped, making it easier to explain. However, the actual implementation uses type synonym families, which is makes it clearer how the type should be used and has better error messages.

The standalone definition uses a \inlinehaskell{data family}. Then, we want to write for every representation type an instance on how to represent the representation type in the context. 

\begin{minted}{haskell}
data family Ctx (f :: * -> *) :: * -> *
\end{minted}

The \texttt{K} and \texttt{U} representation-types do not have a datatype, because these representation-types cannot be traversed through.

\begin{minted}{haskell}
data instance Ctx (K a) r
data instance Ctx U     r
\end{minted}

The sum representation-type does get traversed, but only has one choice by either traversing the left \inlinehaskell{CL} or the right \inlinehaskell{CR} side.

\begin{minted}{haskell}
data instance Ctx (f :+: g) r = CL (Ctx f r) | CR (Ctx g r)
\end{minted}

The product representation-type does have a choice between two traversals, either traverse through the left side and store the right side or traverse through the right side and store the left side.

\begin{minted}{haskell}
data instance Ctx (f :*: g) r = C1 (Ctx f r) (g r) | C2 (f r) (Ctx g r)
\end{minted}

The recursive representation-type does not recursively go into a new \texttt{Ctx}, but into the recursive type \texttt{r}, thus we only need to define datatype which indicates that it is a recursive position.

\begin{minted}{haskell}
data instance Ctx I r = CId
\end{minted}

The \texttt{Zipper} type class can now be defined using the \texttt{Ctx}. There are 6 primary functions, which we need to build navigating functions. The \texttt{cmap} function works like the \texttt{fmap}, but over contexts. The \texttt{fill} function fills the hole in a context with a given value, which is used to reconstruct the data structure. The final 4 functions (\texttt{first}, \texttt{last}, \texttt{next} and \texttt{prev}) are the \textit{primary} navigation operations used to build \textit{interface} functions such as \texttt{left}, \texttt{right}, \texttt{up}, \texttt{down}, etc.

\begin{minted}{haskell}
class Functor f => Zipper f where
  cmap        :: (a -> b) -> Ctx f a -> Ctx f b
  fill        :: Ctx f a -> a -> f a
  first, last :: f a -> Maybe (a, Ctx f a)
  next, prev  :: Ctx f a -> a -> Maybe (a, Ctx f a)
\end{minted}

Finally, we can define the location \texttt{Loc} using the \texttt{Ctx} and the \texttt{Zipper}. The location takes a datatype with the constraints that it has an instance for \texttt{Regular} and a pattern functor which works with the \texttt{Zipper} type class, a list of contexts and returns a location datatype \texttt{Loc}.

\begin{minted}{haskell}
data Loc :: * -> * where
  Loc :: (Regular a, Zipper (PF a)) => a -> [Ctx (PF a) a] -> Loc a
\end{minted}

To use the \texttt{Merkle} type, we need to fulfill the previous constraints. Thus, we need to define a type instance for the pattern functor and an instance for the \texttt{Regular} typeclass. The pattern functor type instance is the same definition as the \texttt{Merkle} type but without the fixed-point. The \texttt{Regular} instance for the \texttt{Merkle} type is folding/unfolding the fixed-point. Moreover, the \texttt{Merkle} type also needs to update the digests of its parents when a value is changed. This is accomplished in the same manner as in Section \ref*{subsec-zipper-treeh}.

\begin{minted}{haskell}
type instance PF (Merkle f) = f :*: K Digest
instance Regular (Merkle f) where
  from = out
  to   = In
\end{minted}