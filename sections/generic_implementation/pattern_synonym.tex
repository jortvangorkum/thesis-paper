\section{Pattern Synonyms}
\label{sec-patt-syn}

The developer experience using \texttt{cataMerkle} is difficult, because the developer needs to know the pattern functor of its datatype to define a function and the function definitions are quite verbose. To make the use of \texttt{cataMerkle} easier, we introduce \textit{pattern synonyms}\cite{pickering2016pattern}. 

Pattern synonyms add an abstraction over patterns, which allows the user to move additional logic from guards and case expressions into patterns. For example, the pattern functor of the \texttt{Tree} datatype can be represented using a \texttt{pattern}.

\begin{minted}{haskell}
pattern Leaf_ :: a -> PF (Tree a) r
pattern Leaf_ x <- L (K x) where
  Leaf_ x = L (K x)

pattern Node_ :: r -> a -> r -> PF (Tree a) r
pattern Node_ l x r <- R (I l :*: K x :*: I r) where
  Node_ l x r = R (I l :*: K x :*: I r)
\end{minted}

The previously defined \texttt{pattern}s can then be used to define the \texttt{cataSum} as the original datatype \texttt{Tree}, but the constructor names leading with an additional underscore. 

\begin{minted}{haskell}
cataSum :: Merkle (PF (Tree Int)) -> (Int, HashMap Digest Int)
cataSum = cataMerkle
  (\case
    Leaf_ x     -> x
    Node_ l x r -> l + x + r
  )
\end{minted}