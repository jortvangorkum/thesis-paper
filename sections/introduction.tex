\chapter{Introduction}

\todo[inline]{Write more context}
\todo[inline]{Add a problem statement}

Incremental computation is an approach to improve performance by computing only the result of the changed parts of the input, instead of computing the entire result. To determine what parts of the input has changed, we provide a \textit{Zipper} which can be used by the user to efficiently update the previous input to a new input and keeps track of the changes to the input. As a result, we know which parts of the input stays the same. Therefore, we can reuse the results of a previously performed computation for the same input\footnote{The technique for reusing results based on input is called \textit{memoization}}.   

The task of implementing incremental computation for Haskell is explained in Chapter \ref*{chap-spec-impl}. However, the task for implementing incremental computation for multiple datatypes, becomes repetitive and error-prone. To prevent writing implementations for every datatype, we use datatype-generic programming. Datatype-generic programming is a technique that uses the structure of a datatype to define functions for a large class of datatypes, which is explained in more detail in Chapter \ref*{chap-dat-gen-program}. Then, using the Haskell generics library \texttt{regular}, we implement the generic version. In Chapter \ref*{chap-gen-impl}, the generic implementation is explained and additionally describe: complexity, different garbage collection strategies and different data structures for storing results. 

To illustrate how the incremental computation functionality is used, an example is presented which implements the \textbf{max-path-sum} in a non-incremental and an incremental manner. The max-path-sum function traverses through a tree and returns the sum of the path with the highest value. For the example, we first define the definition of a tree and an example tree which is used as input for the max-path-sum. Then, the non-incremental function (\inlinehaskell{maxPathSum}) is implemented, by returning the value for the leaf; and for the node recursively calling the children and determine the highest value between the children and adding it to its own value.

\begin{minted}{haskell}
data BinTree = Leaf Int
             | Node BinTree Int BinTree 
             
exampleTree :: BinTree    
exampleTree = Node (Node (Leaf 8) 7 (Leaf 1)) 3 (Node (Leaf 5) 4 (Leaf 2))

-- The non-incremental implementation of max-path-sum
maxPathSum :: BinTree -> Int
maxPathSum (Leaf x)     = x
maxPathSum (Node l x r) = x + max (maxPathSum l) (maxPathSum r)

> maxPathSum exampleTree
    18

> let exampleTree' = update (const (Leaf 6)) [Bttm] exampleTree

> maxPathSum exampleTree'
    16
\end{minted}

The incremental implementation, first adds hashes to the input which represent the internal data structure to the data structure (also known as a \textit{merkle tree}). This is used to efficiently check what parts of the input has changed. Then, the initial computation is performed which gives the result and a map of all the intermediate results. Next, the tree gets updated, in this case the bottom left leaf (\inlinehaskell{Leaf 8}) is replaced with \inlinehaskell{Leaf 6}. Finally, the updated tree is recomputed reusing the previously computed results. As a result, the complete right side of the root node did not have to be recomputed, because it stayed the same.

\todo[inline]{Add the explanation of that we use the Zipper for updating the Tree}

\begin{minted}{haskell}
-- The incremental implementation of max-path-sum
incMaxPathSum :: BinTree -> Int
incMaxPathSum (Leaf_ x)     = x
incMaxPathSum (Node_ l x r) = x + max l r

-- Add hashes to data structure
> let merkleTree = merkle exampleTree

-- Initial computation
> let (y, m) = cataMerkle incMaxPathSum (merkleTree)
    (18, { "6dd": 18, "5df": 15, "fa0": 8, "8d0": 1, "f3b": 9, "84b": 5
         , "1ad": 2 })

-- Update Tree
> let merkleTree' = update (const (merkle (Leaf 6))) [Bttm] merkleTree

-- Incremental computation
> cataMerkleMap incMaxPathSum m (merkleTree')
    (16, { "6dd": 18, "5df": 15, "fa0": 8, "bbd": 16, "91c": 13, "3af": 6
         , "8d0": 1, "f3b": 9, "84b": 5, "1ad": 2 })
\end{minted}

\section{Contributions}

The main contributions of the Thesis are the following:

\begin{itemize}
    \item We define an algorithm for incremental computation over recursive data structures. The algorithm uses hashes for comparing if the data structures are equal in a constant time and a Zipper to efficiently update the recursive data structure without rehashing the entire data structure.
    \item We use datatype-generic programming to write a generic version of the algorithm, to support a large class of datatypes, namely \textit{regular datatypes}.
    \item We use pattern synonyms, to make the developer experience the same as implementing a non-incremental algorithm.
    \item We define cache addition policies and cache replacement policies to optimize the performance/memory usage for different use-cases.
\end{itemize}
