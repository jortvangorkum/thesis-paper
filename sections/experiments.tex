\chapter{Experiments}

\section{Method}

We conducted experiments over three type of functions: the \texttt{Cata Sum}, \texttt{Generic Cata Sum} and \texttt{Incremental Cata Sum}. The Cata Sum is the simple function which traverses through the entire tree and sums all the values. The Generic Cata Sum is the initial Incremental Cata Sum, which starts with an empty Map. And the Incremental Cata Sum, which already has a Map filled with intermediate results and keeps track over multiple iterations.

\begin{minted}{haskell}
cataSum :: Tree Int -> Int
cataSum (Leaf x)     = x
cataSum (Node l x r) = x + cataSum l + cataSum r

genericCataSum :: Merkle (PF (Tree Int)) -> (Int, HashMap Digest Int)
genericCataSum = cataMerkle
  (\case
    Leaf_ x     -> x
    Node_ l x r -> l + x + r
  )

incCataSum :: HashMap Digest Int 
           -> Merkle (PF (Tree Int)) -> (Int, HashMap Digest Int)
incCataSum = cataMerkleMap
  (\case
    Leaf_ x     -> x
    Node_ l x r -> l + x + r
  )
\end{minted}

The experiments will be benchmarks executed with the Haskell package \texttt{criterion}\cite{hackage2022criterion}. Criterion performs the benchmarks multiple times to get an average result. The benchmarks will track two metrics: the execution time and the memory usage. The execution time will be in seconds and the memory usage will be the max-bytes used. The results gather for execution time comes from the criterion package, however the memory usage will not come from the package. This is because criterion only keeps track of the memory allocation and not usage. Therefore, we measure the memory usage with the GHC profiler\cite*{ghc2022memoryprofiling} and profile every benchmark individually to know its memory usage.

To test how well the three functions perform, we perform three types of updates, multiple times. These benchmarks will be based on the three type of cases: worst, average and best.
\begin{enumerate*}[label=(\arabic*)]
  \item The worst case updates the lowest left leaf to a new leaf.
  \item The average case updates a node in the middle with a new leaf.
  \item And the best case replaces the left child of the root-node with a new leaf.
\end{enumerate*}

\begin{minted}{haskell}
worstCase :: Merkle (PF (Tree Int)) -> Merkle (PF (Tree Int))
worstCase = update (const (Leaf i)) [Bttm]

averageCase :: Merkle (PF (Tree Int)) -> Merkle (PF (Tree Int))
averageCase = update (const (Leaf i)) (replicate n Dwn)
  where
    n = round (logBase 2.0 (fromIntegral n) / 2.0)

bestCase :: Merkle (PF (Tree Int)) -> Merkle (PF (Tree Int))
bestCase = update (const (Leaf i)) [Dwn]
\end{minted}

\newpage
\section{Results}

\subsection{Execution Time}

For all the benchmarks, the Incremental Cata Sum is faster when the tree contains more than $10^3$ nodes. However, for every benchmark the execution time is better or worse depending on the type of update that is performed. The best case scenario has the biggest difference, then the average case and the closest performance difference is the worst case. The execution time for Cata Sum and Generic Cata Sum seems to be linear with the amount of nodes and the Incremental Cata Sum is constant/logarithmic. The discrepancy in the range between $10^2 - 10^3$ is probably because the amount of nodes is too low to get stable results.

\begin{figure}[H]
  \begin{minipage}{.5\textwidth}
    \centering
    \includegraphics[width=\textwidth]{plots/run-3/time/Worst/10/all_benchmarks.pdf}  
  \end{minipage}
  \begin{minipage}{.5\textwidth}
    \centering
    \includegraphics[width=\textwidth]{plots/run-3/time/Average/10/all_benchmarks.pdf}  
  \end{minipage}
  \begin{center}
    \begin{minipage}[c]{.5\textwidth}
      \centering
      \includegraphics[width=\textwidth]{plots/run-3/time/Best/10/all_benchmarks.pdf}  
    \end{minipage}
  \end{center}
  \caption{The execution time over 10 executions for the Worst, Average and Best case.}
\end{figure}

\newpage
\subsection{Memory Usage}

The memory usage results are all linear with the amount of nodes. The Cata Sum uses the least amount of memory, then the Incremental Cata Sum and the most used bytes is the Generic Cata Sum. We think that the Incremental Cata Sum uses less memory than the Generic Cata Sum, because the Incremental Cata Sum needs to load-in fewer parts of the data structure than the Generic Cata Sum. Also, the same discrepancy occurs in the range between $10^2 - 10^3$ as with the execution time.

\begin{figure}[H]
  \begin{minipage}{.5\textwidth}
    \centering
    \includegraphics[width=\textwidth]{plots/run-3/memory/Worst/10/all_benchmarks.pdf}  
  \end{minipage}
  \begin{minipage}{.5\textwidth}
    \centering
    \includegraphics[width=\textwidth]{plots/run-3/memory/Average/10/all_benchmarks.pdf}  
  \end{minipage}
  \begin{center}
    \begin{minipage}[c]{.5\textwidth}
      \centering
      \includegraphics[width=\textwidth]{plots/run-3/memory/Best/10/all_benchmarks.pdf}  
    \end{minipage}
  \end{center}
  \caption{The max-bytes-used over 10 executions for the Worst, Average and Best case.}
\end{figure}

\newpage
\subsection{Comparison Cache Addition Policies}

\subsubsection{Minimum Recursion Height of 5}
\begin{figure}[H]
  \begin{minipage}{.5\textwidth}
    \centering
    \includegraphics[width=\textwidth]{plots/run-4/time/Worst/10/all_benchmarks.pdf}  
  \end{minipage}
  \begin{minipage}{.5\textwidth}
    \centering
    \includegraphics[width=\textwidth]{plots/run-4/time/Average/10/all_benchmarks.pdf}  
  \end{minipage}
  \begin{center}
    \begin{minipage}[c]{.5\textwidth}
      \centering
      \includegraphics[width=\textwidth]{plots/run-4/time/Best/10/all_benchmarks.pdf}  
    \end{minipage}
  \end{center}
  \caption{The execution time over 10 executions for the Worst, Average and Best case, where the minimum recursion height is 5.}
\end{figure}
\begin{figure}[H]
  \begin{minipage}{.5\textwidth}
    \centering
    \includegraphics[width=\textwidth]{plots/run-4/memory/Worst/10/all_benchmarks.pdf}  
  \end{minipage}
  \begin{minipage}{.5\textwidth}
    \centering
    \includegraphics[width=\textwidth]{plots/run-4/memory/Average/10/all_benchmarks.pdf}  
  \end{minipage}
  \begin{center}
    \begin{minipage}[c]{.5\textwidth}
      \centering
      \includegraphics[width=\textwidth]{plots/run-4/memory/Best/10/all_benchmarks.pdf}  
    \end{minipage}
  \end{center}
  \caption{The max-bytes-used over 10 executions for the Worst, Average and Best case, where the minimum recursion height is 5.}
\end{figure}

\subsubsection{Minimum Recursion Height of 10}
\begin{figure}[H]
  \begin{minipage}{.5\textwidth}
    \centering
    \includegraphics[width=\textwidth]{plots/run-5/time/Worst/10/all_benchmarks.pdf}  
  \end{minipage}
  \begin{minipage}{.5\textwidth}
    \centering
    \includegraphics[width=\textwidth]{plots/run-5/time/Average/10/all_benchmarks.pdf}  
  \end{minipage}
  \begin{center}
    \begin{minipage}[c]{.5\textwidth}
      \centering
      \includegraphics[width=\textwidth]{plots/run-5/time/Best/10/all_benchmarks.pdf}  
    \end{minipage}
  \end{center}
  \caption{The execution time over 10 executions for the Worst, Average and Best case, where the minimum recursion height is 10.}
\end{figure}
\begin{figure}[H]
  \begin{minipage}{.5\textwidth}
    \centering
    \includegraphics[width=\textwidth]{plots/run-5/memory/Worst/10/all_benchmarks.pdf}  
  \end{minipage}
  \begin{minipage}{.5\textwidth}
    \centering
    \includegraphics[width=\textwidth]{plots/run-5/memory/Average/10/all_benchmarks.pdf}  
  \end{minipage}
  \begin{center}
    \begin{minipage}[c]{.5\textwidth}
      \centering
      \includegraphics[width=\textwidth]{plots/run-5/memory/Best/10/all_benchmarks.pdf}  
    \end{minipage}
  \end{center}
  \caption{The max-bytes-used over 10 executions for the Worst, Average and Best case, where the minimum recursion height is 10.}
\end{figure}